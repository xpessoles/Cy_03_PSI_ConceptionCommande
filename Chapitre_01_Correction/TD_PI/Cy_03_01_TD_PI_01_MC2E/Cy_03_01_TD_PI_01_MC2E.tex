\documentclass[10pt,fleqn]{article} % Default font size and left-justified equations
\usepackage[%
    pdftitle={Résolutions de problèmes de statique : PFS 3D},
    pdfauthor={Xavier Pessoles}]{hyperref}

\input{style/new_style}
\input{style/macros_SII}
\usepackage{multicol}
\usepackage{siunitx}
%\usepackage{picins}
\fichetrue
%\fichefalse

\proftrue
\proffalse

\tdtrue
%\tdfalse

\courstrue
\coursfalse

% -------------------------------------
% Déclaration des titres
% -------------------------------------

\def\discipline{Sciences \\Industrielles de \\ l'Ingénieur}
\def\xxtete{Sciences Industrielles de l'Ingénieur}


\def\classe{\textsf{PSI$\star$ -- MP}}
\def\xxnumpartie{Rév -- Stat}
\def\xxpartie{Modéliser le comportement statique des systèmes mécaniques}

\def\xxnumchapitre{Révision 1 \vspace{.2cm}}
\def\xxchapitre{\hspace{.12cm} Résolution des problèmes de statique -- Statique plane}

\def\xxposongletx{2}
\def\xxposonglettext{1.45}
\def\xxposonglety{19}%16

\def\xxonglet{\textsf{Rév -- Stat}}

\def\xxactivite{TD 01}
\def\xxauteur{\textsl{Xavier Pessoles}}


\def\xxtitreexo{Micromanipulateur compact pour la chirurgie endoscopique ($\text{MC}^2\text{E}$)}
\def\xxsourceexo{\hspace{.2cm} \footnotesize{Concours Commun Mines Ponts 2016}}

\def\xxcompetences{%
\textsl{%
\textbf{Savoirs et compétences :}\\
}}

\def\xxfigures{
\includegraphics[width=.55\textwidth]{images/fig_01}
}%figues de la page de garde

\def\xxpied{%
Révision statique -- Résolution des problèmes de statique plane\\
Fiche 1 -- \xxactivite%
}

\setcounter{secnumdepth}{5}
%---------------------------------------------------------------------------


\begin{document}
%\chapterimage{png/Fond_Cin}
\input{style/new_pagegarde}
\vspace{4.5cm}
\pagestyle{fancy}
\thispagestyle{plain}


\def\columnseprulecolor{\color{ocre}}
\setlength{\columnseprule}{0.4pt} 

\ifprof
\else
\begin{multicols}{2}
\fi
\section*{Mise en situation}
\ifprof
\else
Le robot $\text{MC}^2\text{E}$ est utilisé par des chirurgiens en tant que troisième main lors de l'ablation de la vésicule biliaire. La cinématique du robot permet de garantir que le point d'insertion des outils chirurgicaux soit fixe dans le référentiel du patient. 

Le robot est constitué de 3 axes de rotations permettant de mettre en position une pince. La pince est animée d'un mouvement de translation permettant de tirer la vésicule pendant que le chirurgien la détache du foie. 


\begin{obj}
Valider par un calcul simplifié de pré-dimensionnement la motorisation de l'axe 1 du  $\text{MC}^2\text{E}$.

\end{obj}


\end{multicols}

%%%%%%%%%%%%


\section{Micromanipulateur compact pour la chirurgie endoscopique ($\text{MC}^2\text{E}$)}
\subsection{Présentation générale}
L’objet de cette étude est un robot appelé $\text{MC}^2\text{E}$ utilisé en chirurgie endoscopique. Ce type de
robots médico-chirurgicaux est équipé de capteurs (caméra, capteur d’efforts…) permettant de maîtriser
les interactions avec des environnements souvent déformables et difficilement modélisables comme le
corps humain.

\begin{center}
\begin{minipage}[c]{.1\linewidth}
\begin{center}
\includegraphics[height=4cm]{images/Sujet/images/fig_01}
\end{center}
\end{minipage} \hfill
\begin{minipage}[c]{.65\linewidth}
\begin{center}
\includegraphics[height=3.5cm]{images/Sujet/images/fig_02}
%Figure 1 : Modèle de la commande en effort
\end{center}
\end{minipage}
\end{center}

Le mode opératoire se décompose en quatre phases :
\begin{itemize}
\item phase 1 : après avoir introduit le trocart, l’abdomen du patient est gonflé avec du $\text{CO}_2$. Celui-ci se montrera alors aussi stable et rigide que possible pour la réussite de l’opération ;
\item phase 2 : le $\text{MC}^2\text{E}$ est positionné sur l’abdomen du patient. Celui-ci est maintenu en position grâce à des sangles. Les trois axes en rotation sont alors asservis en position constante ;
\item phase 3 : la pince est introduite dans le trocart au travers d’un guide (étanche). Une phase de calibration du robot utile à la compensation de pesanteur analysée par la suite, démarre ;
\item phase 4 : le chirurgien amène la pince du $\text{MC}^2\text{E}$ qui doit tirer la vésicule lors de l’opération. 
\end{itemize}
L’axe en translation du $\text{MC}^2\text{E}$ entre alors en fonctionnement : il est asservi en effort constant pour tirer (ou pousser) la vésicule au fur et à mesure que le chirurgien utilise son bistouri pour détacher la vésicule du foie. La figure suivante décrit les principales exigences auxquelles est soumis le $\text{MC}^2\text{E}$.

\begin{center}
\includegraphics[width=\linewidth]{images/Sujet/images/fig_05}
%Figure 1 : Modèle de la commande en effort
\end{center}

%L’objectif de ce sujet est d’analyser, de comprendre et de justifier les choix structurels faits par les
%ingénieurs. Pour cela, on se basera sur la démarche de l’ingénieur :
%\begin{itemize}
%\item les exigences et/ou performances souhaitées sont spécifiées tout au long du sujet;
%\item des modèles et résultats analytiques ou simulés sont mis en place;
%\item des résultats expérimentaux sont proposés.
%\end{itemize}
%À chaque fois, on cherchera à quantifier les écarts entre les différents résultats obtenus par simulation et/ou
%expérimentation et les exigences et/ou performances souhaitées. Les réponses apportées aux questions
%devront donc être rédigées dans cet esprit.

\subsection{Validation des performances de l’asservissement d’effort}
On s’intéresse ici à la phase 4. Lors de l’opération envisagée, il est nécessaire de maintenir un effort
constant au bout de la pince (4). Pour cela, on réalise un asservissement d’effort de l’axe en translation que
l’on se propose d’étudier.
Le système est alimenté par un transformateur alternatif/continu. Un variateur permet de piloter le moteur
M4. Une interface de communication entrée/sortie permet de coder les consignes d’effort et acquérir des
grandeurs physiques. D’autre part, elle communique à la chaîne d’énergie, après traitement, des ordres
définis par un calculateur.
La description par diagramme partiel de définition de blocs de l’axe en translation est donnée ci-dessous. %Des modèles géométriques de cet axe sont donnés en annexe 5.


\begin{center}
%\begin{minipage}[c]{.45\linewidth}
\begin{center}
\includegraphics[height=9cm]{images/Sujet/images/fig_03}
\end{center}
%\end{minipage} \hfill
%\begin{minipage}[c]{.45\linewidth}
%\begin{center}
%\includegraphics[height=4cm]{images/Sujet/images/fig_04}
%%Figure 1 : Modèle de la commande en effort
%\end{center}
%\end{minipage}
\end{center}

\subparagraph{}
\textit{Compléter le schéma représentant les chaînes d’énergie et d’information de cette chaîne
fonctionnelle asservie en indiquant le nom des composants réalisant chacune des fonctions.}

%
%
%Lors de l’opération, il est essentiel de contrôler et réguler l’effort appliqué sur l’organe et donc
%indirectement l’effort fourni par le moteur M4. Le schéma-blocs fonctionnel retenu pour la structure
%d’asservissement est donné figure 1. 
%
%\begin{center}
%%\includegraphics[width=\linewidth]{images/Sujet/images/}
%Figure 1 : Modèle de la commande en effort
%\end{center}
%
%A un effort de consigne va correspondre un effort appliqué sur l’organe
%pour l’extraire. C’est ce même effort qui est mesuré par le capteur
%d’effort. Celui-ci va alors générer un couple rapporté sur l’arbre du
%moteur M4.
%On souhaite ici s’intéresser à la structure de commande retenue pour
%cette boucle d’asservissement. Les interactions avec l’organe étant par
%définition inconnues et complexes, on va régler le calculateur en se
%basant sur un montage d’essai mettant en interaction la pince (4) avec
%un ressort simulant la vésicule biliaire (raideur du ressort similaire à la
%raideur de la vésicule).
%
%Le schéma-blocs fonctionnel retenu pour cette étude est donc le
%suivant :
%
%\begin{center}
%%\includegraphics[width=\linewidth]{images/Sujet/images/}
%Figure 2 : Modèle de la commande en effort
%\end{center}


\subsubsection*{Modèle de connaissance de l'asservissement}

\begin{obj}
Modéliser l’asservissement en effort.
\end{obj}

L’équation de mouvement est définie par l’équation différentielle suivante : 
$J\dfrac{\text{d}^2\theta_m(t)}{\text{d}t^2}=C_m(t)-C_e(t)$  avec :
\begin{itemize}
\item $J$, inertie équivalente à l’ensemble en mouvement, ramenée sur l’arbre moteur;
\item $C_e(t)$, couple regroupant l’ensemble des couples extérieurs ramenés à l’arbre moteur, notamment fonction de la raideur du ressort.
\end{itemize}


On notera $\theta_m(p)$, $\Omega_m(p)$, $C_m(p)$ et $C_e(p)$ les transformées de Laplace des grandeurs de l’équation de mouvement.
On pose $C_e(t)=K_{C\theta}\theta_m(t)$ où  $K_{C\theta}$ est une constante positive. On a de plus $\dfrac{\text{d}\theta_m(t)}{\text{d}t}=\omega_m(t)$. La régulation se met alors sous la forme du schéma-blocs à retour unitaire simplifié que l’on
admettra :

\begin{center}
\includegraphics[width=.8\linewidth]{images/Sujet/images/fig_06}

\textit{Modèle simplifié du montage du capteur d’effort.}
\end{center}
%\end{multicols}

Avec :
\begin{itemize}
\item $C_e(p)$, couple de sortie mesuré par le capteur d’effort situé sur le $\text{MC}^2\text{E}$ ;
\item $C_c(p)$, couple de consigne ;
\item $C_m(p)$, couple moteur ;
\item $H_{\text{cor}}(p)$, fonction de transfert du correcteur.
\end{itemize}
Dans un premier temps, on prendra $H_{\text{cor}}(p)=1$.

\subparagraph{}
\textit{Déterminer les expressions des fonctions de transfert $H_1(p)$, $H_2(p)$ et $H_3(p)$.}
\ifprof
\begin{corrige}
\end{corrige}
\else
\fi

\subparagraph{}
\textit{Donner l’expression de la fonction de transfert en boucle fermée $H_{BF}(p)$ de l’asservissement
d’effort.}

\ifprof
\begin{corrige}
\end{corrige}
\else
\fi

\subparagraph{}
\textit{Quel sera le comportement de cet asservissement en réponse à un échelon d'amplitude $C_0$?
Conclure.}
\ifprof
\begin{corrige}
\end{corrige}
\else
\fi

\vspace{.25cm}

Pour remédier au problème ainsi mis en évidence, le concepteur a choisi de mettre en place une boucle
interne numérique, dite tachymétrique, de gain $B$. On s’intéresse ici à la définition analytique de $B$.
Le schéma-blocs modifié est donné figure suivante.


\begin{center}
\includegraphics[width=.7\linewidth]{images/Sujet/images/fig_07}

\textit{Régulation avec retour tachymétrique}
\end{center}


On règle B de telle façon que, pour $H_{\text{cor}}(p)=1$, la fonction de transfert en boucle ouverte, notée $H_{\text{BO}}(p)$, puisse être mise sous la forme suivante : 
$H_{\text{BO}}(p)=\dfrac{1}{\left(1+\tau p\right)^2}$.



\subparagraph{}
\textit{Donner l’expression analytique du gain $B$, en fonction de $J$ et $K_{C\theta}$, permettant d’obtenir cette
forme de fonction de transfert. En déduire l’expression analytique de la constante de temps $\tau$.}
\ifprof
\begin{corrige}
\end{corrige}
\else
\fi

\vspace{.25cm}

Les exigences du cahier des charges sont données plus haut (exigences 1.2.2.1 à 1.2.2.4).

Afin de répondre à ces exigences, on choisit un correcteur proportionnel-intégral de gain $K_i$ et de constante de temps $T_i$. Le schéma-blocs de la régulation se met sous la forme de la figure qui suit.

\begin{center}
\includegraphics[width=.55\linewidth]{images/Sujet/images/fig_08}

\textit{Régulation avec correcteur PI.}
\end{center}


\subparagraph{}
\textit{Donner l’expression de l’erreur statique en réponse à un échelon d'amplitude $C_0$. Conclure vis-à-vis du cahier des charges.}
\ifprof
\begin{corrige}
\end{corrige}
\else
\fi

\vspace{.25cm}

On souhaite régler le correcteur pour que le système asservi ait une fonction de transfert en boucle fermée
d’ordre 2 de la forme :
$\dfrac{K_{\text{BF}}}{1+\dfrac{2\xi_{BF}}{\omega_{0\text{BF}}}p+\dfrac{p^2}{\omega_{0\text{BF}}^2}}$.


\subparagraph{}
\textit{Proposer une expression simple pour la constante de temps $T_i$.}
\ifprof
\begin{corrige}
\end{corrige}
\else
\fi

Sur le document réponse sont tracées les courbes de la réponse fréquentielle en boucle ouverte pour
$K_i=1$ et les réponses fréquentielles en boucle fermée pour différentes valeurs de $K_i$.


\subparagraph{}
\textit{En reportant les tracés nécessaires sur le document réponse et en utilisant les abaques 1 et 2 du
document réponse, proposer un choix de réglage pour $K_i$ permettant de vérifier toutes les
performances.}
\ifprof
\begin{corrige}
\end{corrige}
\else
\fi


\subparagraph{}
\textit{Remplir le tableau du document réponse et conclure sur la validation des critères de performance.
Tracer l’allure de la réponse temporelle à un échelon $C_{c0}$ en indiquant toutes les valeurs caractéristiques
nécessaires.}

\ifprof
\begin{corrige}

\end{corrige}
\else
\fi

%%%%%%%%%%%%%%
\end{document}

\subparagraph{}\textit{}

\begin{center}
\includegraphics[width=\linewidth]{images/img_04}
%\textit{}
\end{center}

