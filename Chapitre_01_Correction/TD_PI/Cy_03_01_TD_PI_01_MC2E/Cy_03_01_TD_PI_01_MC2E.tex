\documentclass[10pt,fleqn]{article} % Default font size and left-justified equations
\usepackage[%
    pdftitle={Résolutions de problèmes de statique : PFS 3D},
    pdfauthor={Xavier Pessoles}]{hyperref}

\input{style/new_style}
\input{style/macros_SII}
\usepackage{multicol}
\usepackage{siunitx}
%\usepackage{picins}
\fichetrue
%\fichefalse

\proftrue
%\proffalse

\tdtrue
%\tdfalse

\courstrue
\coursfalse

% -------------------------------------
% Déclaration des titres
% -------------------------------------

\def\discipline{Sciences \\Industrielles de \\ l'Ingénieur}
\def\xxtete{Sciences Industrielles de l'Ingénieur}


\def\classe{\textsf{PSI$\star$ -- MP}}
\def\xxnumpartie{Rév -- Stat}
\def\xxpartie{Modéliser le comportement statique des systèmes mécaniques}

\def\xxnumchapitre{Révision 1 \vspace{.2cm}}
\def\xxchapitre{\hspace{.12cm} Résolution des problèmes de statique -- Statique plane}

\def\xxposongletx{2}
\def\xxposonglettext{1.45}
\def\xxposonglety{19}%16

\def\xxonglet{\textsf{Rév -- Stat}}

\def\xxactivite{TD 01}
\def\xxauteur{\textsl{Xavier Pessoles}}


\def\xxtitreexo{Micromanipulateur compact pour la chirurgie endoscopique ($\text{MC}^2\text{E}$)}
\def\xxsourceexo{\hspace{.2cm} \footnotesize{Concours Commun Mines Ponts 2016}}

\def\xxcompetences{%
\textsl{%
\textbf{Savoirs et compétences :}\\
}}

\def\xxfigures{
\includegraphics[width=.55\linewidth]{images/fig_01}
}%figues de la page de garde

\def\xxpied{%
Révision statique -- Résolution des problèmes de statique plane\\
Fiche 1 -- \xxactivite%
}

\setcounter{secnumdepth}{5}
%---------------------------------------------------------------------------


\begin{document}
%\chapterimage{png/Fond_Cin}
\input{style/new_pagegarde}
\vspace{4.5cm}
\pagestyle{fancy}
\thispagestyle{plain}


\def\columnseprulecolor{\color{ocre}}
\setlength{\columnseprule}{0.4pt} 

\ifprof
\else
%\begin{multicols}{2}
\fi
\begin{multicols}{2}
\section*{Mise en situation}
\ifprof
\else
Le robot $\text{MC}^2\text{E}$ est utilisé par des chirurgiens en tant que troisième main lors de l'ablation de la vésicule biliaire. La cinématique du robot permet de garantir que le point d'insertion des outils chirurgicaux soit fixe dans le référentiel du patient. 

Le robot est constitué de 3 axes de rotations permettant de mettre en position une pince. La pince est animée d'un mouvement de translation permettant de tirer la vésicule pendant que le chirurgien la détache du foie. 

\textbf{L’axe en translation du $\text{MC}^2\text{E}$ est asservi en effort constant pour tirer (ou pousser) la vésicule au fur et à mesure que le chirurgien utilise son bistouri pour détacher la vésicule du foie. Le diagramme des exigences au dos décrit les principales exigences auxquelles est soumis le $\text{MC}^2\text{E}$.}


\begin{obj}
Modéliser et valider l’asservissement en effort.
\end{obj}

\fi


\subsection*{Modèle de connaissance de l'asservissement}
\ifprof
\else
L’équation de mouvement est définie par l’équation différentielle suivante : 
$J\dfrac{\text{d}^2\theta_m(t)}{\text{d}t^2}=C_m(t)-C_e(t)$  avec :
\begin{itemize}
\item $J$, inertie équivalente à l’ensemble en mouvement, ramenée sur l’arbre moteur;
\item $C_e(t)$, couple regroupant l’ensemble des couples extérieurs ramenés à l’arbre moteur, notamment fonction de la raideur du ressort.
\end{itemize}


On notera $\theta_m(p)$, $\Omega_m(p)$, $C_m(p)$ et $C_e(p)$ les transformées de Laplace des grandeurs de l’équation de mouvement.
On pose $C_e(t)=K_{C\theta}\theta_m(t)$ où  $K_{C\theta}$ est une constante positive. On a de plus $\dfrac{\text{d}\theta_m(t)}{\text{d}t}=\omega_m(t)$. La régulation se met alors sous la forme du schéma-blocs à retour unitaire simplifié que l’on
admettra :

\begin{center}
\includegraphics[width=\linewidth]{images/Sujet/images/fig_06}

\textit{Modèle simplifié du montage du capteur d’effort.}
\end{center}
%\end{multicols}

Avec :
\begin{itemize}
\item $C_e(p)$, couple de sortie mesuré par le capteur d’effort situé sur le $\text{MC}^2\text{E}$ ;
\item $C_c(p)$, couple de consigne ;
\item $C_m(p)$, couple moteur ;
\item $H_{\text{cor}}(p)$, fonction de transfert du correcteur.
\end{itemize}
Dans un premier temps, on prendra $H_{\text{cor}}(p)=1$.

\fi
\subparagraph{}
\textit{Déterminer les expressions des fonctions de transfert $H_1(p)$, $H_2(p)$ et $H_3(p)$.}
\ifprof
\begin{corrige}
On a $p\theta_m(p)=\Omega_m(p)$ et donc $H_2(p)=\dfrac{\theta_m(p)}{\Omega_m(p)}=\dfrac{1}{p}$.

De plus $Jp^2 \theta_m(p) = C_m(p)-C_e(p) \Leftrightarrow Jp\Omega_m(p) = \Omega_m(p)$ et donc $H_1(p)=\dfrac{\Omega_m(p)}{C_m(p)-C_e(p)}=\dfrac{1}{Jp}$.

Enfin, $H_3(p)=\dfrac{C_e(p)}{\theta_m(p)}=K_{C\theta}$.

\end{corrige}
\else
\fi

\subparagraph{}
\textit{Donner l’expression de la fonction de transfert en boucle fermée $H_{\text{BF}}(p)$ de l’asservissement d’effort.}

\ifprof
\begin{corrige}
D'une part, $F(p)=\dfrac{H_1(p)H_2(p)H_3(p)}{1+H_1(p)H_2(p)H_3(p)} =\dfrac{\dfrac{1}{Jp}\dfrac{1}{p}K_{C\theta}}{1+\dfrac{1}{Jp}\dfrac{1}{p}K_{C\theta}} =\dfrac{K_{C\theta}}{Jp^2+K_{C\theta}}$.

D'autre part, $H_{\text{BF}}(p)
=\dfrac{\dfrac{K_{C\theta}}{Jp^2+K_{C\theta}}}{1+\dfrac{K_{C\theta}}{Jp^2+K_{C\theta}}}=\dfrac{K_{C\theta}}{Jp^2+2K_{C\theta}}$.

\end{corrige}
\else
\fi

\subparagraph{}
\textit{Quel sera le comportement de cet asservissement en réponse à un échelon d'amplitude $C_0$?
Conclure.}
\ifprof
\begin{corrige}
Il s'agit d'un système du second ordre avec un coefficient d'amortissement nul. Le gain est de $\dfrac{1}{2}$ et la pulsation est de $\dfrac{1}{\omega_0^2}=\dfrac{J}{2K_{C\theta}} \Rightarrow \omega_0=\sqrt{\dfrac{2K_{C\theta}}{J}}$.
\end{corrige}
\else
\fi


%\vspace{.25cm}
\ifprof
\else
Pour remédier au problème ainsi mis en évidence, le concepteur a choisi de mettre en place une boucle
interne numérique, dite tachymétrique, de gain $B$. On s’intéresse ici à la définition analytique de $B$.
Le schéma-blocs modifié est donné figure suivante.


\begin{center}
\includegraphics[width=\linewidth]{images/Sujet/images/fig_07}

\textit{Régulation avec retour tachymétrique}
\end{center}


On règle B de telle façon que, pour $H_{\text{cor}}(p)=1$, la fonction de transfert en boucle ouverte, notée $H_{\text{BO}}(p)$, puisse être mise sous la forme suivante : 
$H_{\text{BO}}(p)=\dfrac{1}{\left(1+\tau p\right)^2}$.

\fi

\subparagraph{}
\textit{Donner l’expression analytique du gain $B$, en fonction de $J$ et $K_{C\theta}$, permettant d’obtenir cette
forme de fonction de transfert. En déduire l’expression analytique de la constante de temps $\tau$.}
\ifprof
\begin{corrige}~\\

D'une part, $F_1(p)=\dfrac{H_1(p)}{1+H_1(p)B}$. 

D'autre part, 
$H_{\text{BO}}(p)$
$=\dfrac{\dfrac{H_1(p)}{1+H_1(p)B}H_2(p)H_3(p)}{1+\dfrac{H_1(p)}{1+H_1(p)B}H_2(p)H_3(p)}$ 
%$H_{\text{BO}}(p)
$=\dfrac{H_1(p)H_2(p)H_3(p)}{1+H_1(p)B+H_1(p)H_2(p)H_3(p)}$
$=\dfrac{\dfrac{K_{C\theta}}{Jp^2}}{1+\dfrac{B}{Jp}+\dfrac{K_{C\theta}}{Jp^2}}$
$=\dfrac{K_{C\theta}}{Jp^2+Bp+K_{C\theta}}$
$=\dfrac{1}{\dfrac{J}{K_{C\theta}}p^2+\dfrac{B}{K_{C\theta}}p+1}$.

Enfin, $\left(1+\tau p\right)^2 = 1+2\tau p + \tau^2p^2$. Donc nécessairement $\tau^2=\dfrac{J}{K_{C\theta}} \Rightarrow \tau=\sqrt{\dfrac{J}{K_{C\theta}}}$ et 
$2\tau = \dfrac{B}{K_{C\theta}} \Leftrightarrow B
=2\tau K_{C\theta}
=2\sqrt{\dfrac{J}{K_{C\theta}}} K_{C\theta}
=2\sqrt{JK_{C\theta}}$.
\end{corrige}
\else
\fi

%\vspace{.25cm}

\ifprof
\else
Les exigences du cahier des charges sont données plus loin (exigences 1.2.2.1 à 1.2.2.4).

Afin de répondre à ces exigences, on choisit un correcteur proportionnel-intégral de gain $K_i$ et de constante de temps $T_i$. Le schéma-blocs de la régulation se met sous la forme de la figure qui suit.

\begin{center}
\includegraphics[width=\linewidth]{images/Sujet/images/fig_08}

\textit{Régulation avec correcteur PI.}
\end{center}
\fi

\subparagraph{}
\textit{Donner l’expression de l’erreur statique en réponse à un échelon d'amplitude $C_0$. Conclure vis-à-vis du cahier des charges.}
\ifprof
\begin{corrige}
La boucle ouverte est de classe 1. L'erreur statique (entrée échelon) est donc nulle ce qui est conforme à l'exigence 1.2.2.1 du cahier des charges. 
\end{corrige}
\else
\fi

%\vspace{.25cm}

\ifprof
\else
On souhaite régler le correcteur pour que le système asservi ait une fonction de transfert en boucle fermée
d’ordre 2 de la forme :
$\dfrac{K_{\text{BF}}}{1+\dfrac{2\xi_{BF}}{\omega_{0\text{BF}}}p+\dfrac{p^2}{\omega_{0\text{BF}}^2}}$.
\fi

\subparagraph{}
\textit{Proposer une expression simple pour la constante de temps $T_i$.}
\ifprof
\begin{corrige} ~\\

Pour $\tau=T_i$, on a $\dfrac{C_e(p)}{C_C(p)}$
$=\dfrac{\dfrac{K_i}{\tau p\left(1+\tau p \right)}}{1+\dfrac{K_i}{\tau p\left(1+\tau p \right)}}$
$=\dfrac{K_i}{\tau p\left(1+\tau p \right)+ K_i}$
$=\dfrac{K_i}{\tau^2 p^2+\tau p+ K_i}$
$=\dfrac{1}{\dfrac{\tau^2}{K_i} p^2+\dfrac{\tau}{K_i} p+ 1}$.
\end{corrige}
\else
\fi


\ifprof
\else
Les courbes de la réponse fréquentielle en boucle ouverte pour
$K_i=1$ et les réponses fréquentielles en boucle fermée pour différentes valeurs de $K_i$ sont données ci-dessous.
\fi

\subparagraph{}
\textit{En s'appuyant sur les diagrammes ci-dessous, proposer un choix de réglage pour $K_i$ permettant de vérifier toutes les performances.}
\ifprof
\begin{corrige}
\end{corrige}
\else
\fi



\ifprof
\else

\begin{center}
\includegraphics[width=\linewidth]{images/bo}
%\textit{}
\end{center}

\begin{center}
\includegraphics[width=\linewidth]{images/bf}
%\textit{}
\end{center}

\begin{center}
\includegraphics[width=\linewidth]{images/abaque}
%\textit{}
\end{center}

\begin{center}
\includegraphics[width=\linewidth]{images/abaque_tr}
%\textit{}
\end{center}

\fi

\subsection*{Retour sur le cahier des charges}
\subparagraph{}
\textit{Remplir le tableau et conclure sur la validation des critères de performance.
Tracer l’allure de la réponse temporelle à un échelon $C_{c0}$ en indiquant toutes les valeurs caractéristiques nécessaires.}

\begin{center}
\begin{tabular}{|c|c|}
\hline
Critère & Valeur \\ \hline
Marges de gain &  \\ \hline
Marges de phase &  \\ \hline
Dépassement &  \\ \hline
T5~\% &  \\ \hline
Erreur statique & \\ \hline
\end{tabular}
\end{center}
\ifprof
\begin{corrige}

\end{corrige}
\else
\fi


%%%%%%%%%%%%%%
\end{multicols}


\ifprof
\else

\begin{center}
%\includegraphics[width=\linewidth]{images/Sujet/images/fig_05}
\includegraphics[width=\linewidth]{images/mc2e}
%Figure 1 : Modèle de la commande en effort
\end{center}
\fi

\end{document}

\subparagraph{}\textit{}

\begin{center}
\includegraphics[width=\linewidth]{images/img_04}
%\textit{}
\end{center}

