\documentclass[10pt,fleqn]{article} % Derfault font size and left-justified equations
\usepackage[%
    pdftitle={Correction des SLCI : Rapidité des systèmes},
    pdfauthor={Xavier Pessoles}]{hyperref}
    
\input{style/new_style}
\input{style/macros_SII}
\usepackage{multicol}
\usepackage{siunitx}
%\usepackage{picins}
\fichetrue
%\fichefalse

\proftrue
%\proffalse

\tdtrue
%\tdfalse

\courstrue
\coursfalse

\def\discipline{Sciences \\Industrielles de \\ l'Ingénieur}
\def\xxtete{Sciences Industrielles de l'Ingénieur}

\def\classe{PSI$\star$ -- MP}
\def\xxnumpartie{Cycle 03}
\def\xxpartie{Concevoir la partie commande des systèmes asservis afin de valider leurs performances}

\def\xxnumchapitre{Chapitre 1 \vspace{.2cm}}
\def\xxchapitre{\hspace{.12cm} Correction des SLCI}


\def\xxtitreexo{Système de freinage d'un TGV DUPLEX}
\def\xxsourceexo{\hspace{.2cm} \footnotesize{Centrale Supelec -- PSI -- 2006}}


\def\xxposongletx{2}
\def\xxposonglettext{1.45}
\def\xxposonglety{20}
%\def\xxonglet{Part. 1 -- Ch. 3}
\def\xxonglet{\textsf{Cycle 03}}

\def\xxactivite{TD 99}
\def\xxauteur{\textsl{Xavier Pessoles}}

\def\xxcompetences{%
\textsl{%
\textbf{Savoirs et compétences :}\\
\footnotesize{}%Le modèle du système est donné. Des conditions sont estimées sur un paramètre de la FTBO à partir des exigences du cahier des charges. Une \textbf{démarche de réglage d'un correcteur à avance de phase} est appliquée. Une conclusion est menée en déterminant pour chaque exigence l'écart entre la performance simulée et celle attendue.}
%Les sources sont associées par un \emph{hacheur série}. La détermination des grandeurs électriques associées à ce montage permet de conclure vis à vis du cahier des charges.
%\noindent \textbf{Résoudre :} à partir des modèles retenus :
%\begin{itemize}[label=\ding{112},font=\color{ocre}] 
%\item choisir une méthode de résolution analytique, graphique, numérique;
%\item mettre en \oe{}uvre une méthode de résolution.
%\end{itemize}
%\begin{itemize}[label=\ding{112},font=\color{ocre}] 
%\item \textit{Rés -- C1.1 :} Loi entrée sortie géométrique et cinématique -- Fermeture géométrique.
%\end{itemize}
%
%\noindent \textit{Mod2 -- C4.1 :} Représentation par schéma bloc.
}}

\def\xxfigures{
\includegraphics[width=.7\linewidth]{images/fig_00}
}%figues de la page de garde


\def\xxpied{%
Cycle 03 -- Concevoir la partie commande des SLCI \\
Chapitre 1 -- \xxactivite%
}

\setcounter{secnumdepth}{5}
%---------------------------------------------------------------------------

\usepackage{pgfplots}
\begin{document}

%\chapterimage{png/Fond_Cin}
\input{style/new_pagegarde}
\vspace{5cm}
\pagestyle{fancy}
\thispagestyle{plain}

\def\columnseprulecolor{\color{ocre}}
\setlength{\columnseprule}{0.4pt} 

\def\pathfig{images}

\begin{multicols}{2}

\subsection*{Dispositif d'anti-enrayage}
L’objectif de cette partie est l’étude de la loi de commande du dispositif d’anti-enrayage
et plus précisément le calcul du correcteur de la boucle de régulation en
vue de satisfaire un cahier des charges qui sera exprimé par la suite.

La réalisation de la régulation de glissement ne peut être effectuée directement,
en particulier la seule mesure généralement disponible est celle de la vitesse $V_R$, aussi la vitesse $V_T$ est obtenue par estimation. En « pratique », la mise en
place de la chaîne de régulation du dispositif d’anti-enrayage du système de freinage
est conçue de la façon suivante :
\begin{itemize}
\item elle est réalisée au travers de l’asservissement des vitesses des roues à une
consigne de référence obtenue à partir de $V_T$;
\item la commande de l’actionneur est non linéaire, de type tout ou rien ;
\item les algorithmes implémentés visent à optimiser le point de fonctionnement
en vue de minimiser la distance de freinage.
\end{itemize}
Cependant, dans le cadre de cette étude, on supposera que :
\begin{itemize}
\item les vitesses $V_R$ et $V_T$ sont directement accessibles à la mesure, éventuellement
entachées d’une erreur;
\item la régulation peut se ramener directement à celle du glissement;
\item  le comportement de l’actionneur et de sa « commande rapprochée » est modélisé
par une fonction de transfert linéaire correspondant à un comportement
« moyen ».
\end{itemize}
On suppose, pour la suite, que l’architecture de la boucle de régulation est représentée sur la figure suivante où est la consigne de glissement.

\begin{center}
\includegraphics[width=\linewidth]{images/fig_02}
%%\textit{}
\end{center}

\begin{itemize}
\item $H_1(p)$ : fonction de transfert de l’actionneur de freinage (vérin pneumatique
+ électrovalve);
\item $H_2(p)$ : fonction de transfert de la roue au freinage;
\item $C(p)$ : correcteur de la boucle de régulation;
\item $M(p)$ : fonction de transfert de la chaîne de mesure du glissement obtenu à
partir des vitesses $V_T$ et $V_R$, cette chaîne comporte un filtre destiné à limiter
l’impact des bruits de mesure;
\item $\nu_m$ : glissement estimé à partir de $V_T$ et de $V_R$.
\end{itemize}

On adoptera pour la suite : 
$$
H_1(p)=\dfrac{2000}{1+0,1p+0,01p^2}
\quad
M(p)=\dfrac{1}{1+0,05 p}.
$$

Pour une vitesse $V_T=\SI{200}{km.h^{-1}}$, le cahier des charges est résumé par les données 
du tableau suivant et les données numériques utilisées sont données ci-dessous.
Enfin, les problèmes liés à l’évolution de la vitesse $V_T$ ne font pas l’objet de cette
étude.

On note $M=\SI{8200}{kg}$, $V_T=\SI{200}{km.h^{-1}}$, $\dfrac{I}{r^2}=\SI{400}{kg}$, $g=\SI{10}{m.s^{-2}}$.

\subsection*{Analyse des réponses fréquentielles en boucle ouverte}

On donne $H_2(p)=\dfrac{r^2/\left( IV_{T0}\right)}{p} = \dfrac{45\cdot 10^{-6}}{p}$.

\subparagraph{}
\textit{En prenant $C(p)=1$, compléter par le tracé asymptotique le diagramme
de Bode de la fonction de transfert en boucle ouverte fourni en figure suivante
en justifiant le tracé.}

\begin{center}
\includegraphics[width=\linewidth]{images/fig_03}
%\textit{}
\end{center}

\subsection*{Synthèse du régulateur de la boucle de régulation}
On décide d’implémenter un régulateur de type P.I. dont la fonction de transfert
est : $C(p)=K_R\left( 1+\dfrac{1}{T_i p}\right)$.


\subparagraph{}\textit{Calculer la valeur que doit prendre l’argument de $C(p)$ afin d’assurer
la marge de phase imposée par le cahier des charges à la pulsation de coupure $\omega_c$
souhaitée.}

\subparagraph{}\textit{Calculer la valeur minimale $T_{i\text{min}}$ , que l’on peut conférer à la constante $T_i$ de l’action intégrale du régulateur.}


\subparagraph{}\textit{En adoptant $T_i=T_{i\text{min}}$, déterminer alors le gain $K_r$ du régulateur permettant de satisfaire la pulsation de coupure et la marge de phase souhaitées.
}

IV.B.4) Le système étant bouclé par le régulateur dimensionné à la question
IV.B.3, déterminer la marge de gain. Conclure sur les marges de stabilité obtenues.
\subparagraph{}\textit{}

\subparagraph{}\textit{}

\subparagraph{}\textit{}

\subparagraph{}\textit{}
%
%\subparagraph{}\textit{}

\end{multicols}

%
%\begin{center}
%\includegraphics[width=\linewidth]{images/fig_07}
%%\textit{}
%\end{center}


\end{document}

\subparagraph{}\textit{}
\begin{center}
\includegraphics[width=\linewidth]{images/fig_07}
%\textit{}
\end{center}


%\newpage

%\begin{center}
%\includegraphics[width=\linewidth]{images/cor_01}
%\textit{}
%\end{center}






\begin{center}
\includegraphics[width=\linewidth]{images/fig_06}
%\textit{}
\end{center}
\begin{center}
\includegraphics[width=\linewidth]{images/img_04}
%\textit{}
\end{center}

