\documentclass[10pt,fleqn]{article} % Default font size and left-justified equations
\usepackage[%
    pdftitle={Correction des SLCI},
    pdfauthor={Xavier Pessoles}]{hyperref}

\input{style/new_style}
\input{style/macros_SII}

\fichetrue
\fichefalse

\proftrue
%\proffalse

%\tdtrue
\tdfalse

\courstrue
%\coursfalse



% -------------------------------------
% Déclaration des titres
% -------------------------------------

\def\discipline{Sciences \\Industrielles de \\ l'Ingénieur}
\def\xxtete{Sciences Industrielles de l'Ingénieur}

\def\classe{\textsf{PSI$\star$ -- MP}}
\def\xxnumpartie{Cycle 03}
\def\xxpartie{Concevoir la partie commande des systèmes asservis afin de valider leurs performances.}

\def\xxnumchapitre{Chapitre 1 \vspace{.2cm}}
\def\xxchapitre{\hspace{.12cm} Correction des SLCI}

\def\xxposongletx{2}
\def\xxposonglettext{1.45}
\def\xxposonglety{19}%16

\def\xxonglet{Cycle 03}

\def\xxactivite{Cours}
\def\xxauteur{\textsl{Xavier Pessoles}}

\def\xxcompetences{%
\textsl{%
\textbf{Savoirs et compétences :}\\
\begin{itemize}[label=\ding{112},font=\color{ocre}] 
\item \textit{Res1.C4 : } Correction
\item \textit{Res1.C4.SF1 : } Proposer la démarche de réglage d’un correcteur proportionnel, proportionnel intégral et à avance de phase
\item \textit{Con.C2 : } 	Correction d’un système asservi	
\item \textit{Con.C2.SF1 : } Choisir un type de correcteur adapté
\end{itemize}
}}
		
\def\xxfigures{
%\includegraphics[width=1.4\textwidth]{images/matlab}%images/prot_01
%\\
%\textit{Modèle du pilote hydraulique avec pilotage interactif.}
}%figues de la page de garde

\def\xxpied{%
Cycle 03 -- Concevoir la partie commande des systèmes asservis afin de valider leurs performances.\\
Chapitre 1 -- \xxactivite%
}

\setcounter{secnumdepth}{5}
%---------------------------------------------------------------------------


\begin{document}
\chapterimage{png/Fond_SLCI}
\input{style/new_pagegarde}
\setlength{\columnseprule}{.1pt}

\vspace{2cm}
\pagestyle{fancy}
\thispagestyle{plain}
\section{Pourquoi corriger un système ?}

\noindent
\begin{minipage}[c]{.6\linewidth}
\hspace{.4cm} Souvent évoqué en lors de l'étude des systèmes asservis, regardons ce qui se cache derrière le bloc correcteur. On peut le considérer comme la partie intelligente du système car de sa part position dans l'architecture d'un système il reçoit l'image de l'écart entre la cosigne et la sortie du système. En fonction de cet écart, en fonction de ses << capacités >> va permettre d'améliorer les performances du système. 
\end{minipage} \hfill
\begin{minipage}[c]{.35\linewidth}
\begin{center}
\includegraphics[width=\linewidth]{images/fig_01}
\end{center}
\end{minipage} 


\vspace{.25cm}

\noindent
\begin{minipage}[c]{.6\linewidth}
\hspace{.4cm} Sur la figure ci-contre est tracée en gris la réponse indicielle d'un système non corrigé et en noir la réponse indicielle du système corrigé. On observe que le système corrigé est :
\begin{itemize}
\item plus précis;
\item plus amorti;
\item plus rapide. 
\end{itemize}
L'objectif du correcteur est donc d'améliorer les caractéristiques tout en assurant la stabilité su système.
\end{minipage} \hfill
\begin{minipage}[c]{.35\linewidth}
\begin{center}
\includegraphics[width=\linewidth]{images/fig_03}
\end{center}
\end{minipage} 

\begin{resultat} ~\\

\begin{itemize}
\item D'après les résultats sur la stabilité des systèmes asservis :
\begin{itemize}
\item le correcteur doit permettre d'avoir des marges de gains suffisantes.
\end{itemize}
\item D'après les résultats sur la rapidité des systèmes asservis :
\begin{itemize}
\item le correcteur doit permettre d'augmenter le gain dans le but d'avoir une pulsation de coupure à \SI{0}{dB} la plus grande possible.
\end{itemize}
\item D'après les résultats sur la précision des systèmes asservis :
\begin{itemize}
\item le correcteur doit permettre d'augmenter le gain statique de la boucle ouverte pour assurer une bonne précision du système (et d’éventuellement augmenter la classe).
\end{itemize}
\end{itemize}

Au vue de ces conclusions, le choix d'un correcteur se fera dans le domaine fréquentiel en utilisant le diagramme de Bode. 
\end{resultat}

\newpage

\section{Le correcteur proportionnel}

\begin{defi}

Le correcteur proportionnel a pour fonction de transfert $C(p)=K$.

\end{defi}

Prenons le cas d'un système du second ordre bouclé ($K=15$, $\xi=3$, $\omega=1$).

\noindent
\begin{minipage}[c]{.46\linewidth}
\begin{center}
\includegraphics[width=\linewidth]{images/fig_04a}

$\text{T}_{5\%}$ : \SI{0,781}{s} -- Écart statique : 0,07
\end{center}

\end{minipage} \hfill
\begin{minipage}[c]{.46\linewidth}
\begin{center}
\includegraphics[width=\linewidth]{images/fig_04b}

Marge de phase 71,94 \degres
\end{center}
\end{minipage} 


\noindent
\begin{minipage}[c]{.46\linewidth}
\begin{center}
\includegraphics[width=\linewidth]{images/fig_05a}

$\text{T}_{5\%}$ : \SI{0,88}{s} -- Écart statique : tend $\to$ 0
\end{center}

\end{minipage} \hfill
\begin{minipage}[c]{.46\linewidth}
\begin{center}
\includegraphics[width=\linewidth]{images/fig_05b}

Marge de phase 6,43 \degres
\end{center}
\end{minipage} 

\begin{resultat}

On observe qu'une augmentation du gain proportionnel a pour effet :
\begin{itemize}
\item d'améliorer la précision;
\item d'augmenter la vivacité;
\item d'augmenter le temps de réponse (à partir d'un certain seuil);
\item de diminuer l'amortissement;
\item de diminuer la marge de phase.
\end{itemize}
Pour un système d'ordre supérieur à 2, l'augmentation du gain provoque une marge de phase négative et donc une instabilité du système.
\end{resultat}

\section{Le correcteur proportionnel intégral}

\section{Le correcteur à avance de phase}


\begin{thebibliography}{2}
   \bibitem[1]{ref1} Frédéric Mazet, {\it Cours d'automatique de deuxième année, Lycée Dumont Durville, Toulon.}
      \bibitem[2]{ref2} Florestan Mathurin, {\it Correction des SLCI, Lycée Bellevue, Toulouse, \url{http://florestan.mathurin.free.fr/}.}



\end{thebibliography}

\end{document}



