\documentclass[10pt,fleqn]{article} % Derfault font size and left-justified equations
\usepackage[%
    pdftitle={Correction des SLCI : Correcteur PI},
    pdfauthor={Xavier Pessoles}]{hyperref}

\input{style/new_style}
\input{style/macros_SII}
\usepackage{multicol}
\usepackage{siunitx}
\usepackage{schemabloc}
%\usepackage{picins}
\fichetrue
%\fichefalse

\proftrue
\proffalse

\tdtrue
%\tdfalse

\courstrue
\coursfalse

\newif\ifnormal
\normaltrue
%\normalfalse

\newif\ifdifficile
\difficilefalse
%\difficiletrue

\newif\iftdifficile
\tdifficilefalse
%\tdifficiletrue

% -------------------------------------
% Déclaration des titres
% -------------------------------------

\def\classe{\textsf{PSI$\star$ -- MP}}
\def\xxnumpartie{Cycle 03}
\def\xxpartie{Concevoir la partie commande des systèmes asservis afin de valider leurs performances}


\def\xxnumchapitre{Chapitre 1 \vspace{.2cm}}
\def\xxchapitre{\hspace{.12cm} Correction des SLCI}

\def\discipline{Sciences \\Industrielles de \\ l'Ingénieur}
\def\xxtete{Sciences Industrielles de l'Ingénieur}

\def\xxposongletx{2}
\def\xxposonglettext{1.45}
\def\xxposonglety{19}%16

\def\xxonglet{\textsf{Cycle 03}}

\def\xxactivite{TD 99}
\def\xxauteur{\textsl{Xavier Pessoles}}


\def\xxtitreexo{Vision en réalité augmentée pour hélicoptère \ifnormal $\star$ \else \fi \iftdifficile $\star\star\star$ \else \fi }
\def\xxsourceexo{\hspace{.2cm} \footnotesize{Concours Centrale Supelec 2014}}

\def\xxcompetences{%
\textsl{%
\textbf{Savoirs et compétences :}\\
}}

\def\xxfigures{
\includegraphics[width=.65\textwidth]{images/fig_00}
}%figues de la page de garde
\def\xxpied{%
Cycle 03 -- Concevoir la commande des SLCI\\% afin de valider leurs performances.\\
Chapitre 1 -- \xxactivite%
}


\setcounter{secnumdepth}{5}
%---------------------------------------------------------------------------


\begin{document}
%\chapterimage{png/Fond_Cin}
\input{style/new_pagegarde}
\vspace{4.5cm}
\pagestyle{fancy}
\thispagestyle{plain}


\def\columnseprulecolor{\color{ocre}}
\setlength{\columnseprule}{0.4pt} 

\ifprof
%\begin{multicols}{2}
\else
\begin{multicols}{2}
\fi


\section*{Mise en situation}
Le FLIR est une boule optronique modulaire pouvant intégrer plusieurs capteurs différents dont une caméra
thermique, une caméra couleur TV HD, ainsi qu’une caméra très bas niveau de lumière. Cet ensemble est
orientable et gyrostabilisé, c’est-à-dire en particulier que les caméras sont capables de garder une même ligne
de visée par rapport au référentiel terrestre, quels que soient les mouvements de l’hélicoptère NH90 qui sera
appelé porteur dans la suite du sujet.


\begin{center}
\includegraphics[width=\linewidth]{images/fig_01}
%\textit{}
\end{center}

Afin de limiter l’influence des vibrations du porteur sur la ligne de visée et augmenter la précision de son orientation,
les ingénieurs ont choisi de décomposer l’axe motorisé d’élévation en deux étages.
Le premier étage, appelé étage gros d’élévation ($ge$), est en prise directe avec l’air et est donc soumis aux effets
aérodynamiques lors des mouvements du porteur. L’étage gros d’élévation est lui même en liaison pivot, d’axe $\axe{P}{y_e}$, avec l’axe motorisé d’azimut. Le second, appelé étage fin d’élévation ($fe$), est protégé des effets aérodynamiques grâce au carter sphérique solidaire de l’étage gros. Cet étage est en liaison pivot,
d’axe $\axe{P}{y_e}$, avec l’étage gros d’élévation. L’inertie des éléments déplacés par l’étage fin d’élévation est plus faible
que celle de l’étage gros d’élévation et les choix de guidage et de motorisation permettent d’atteindre des accélérations
et des vitesses élevées. Cependant, l’amplitude du mouvement de l’étage fin est limitée. 



Les performances de l’étage fin d’élévation sont données dans le tableau suivant. 
\begin{center}
\begin{tabular}{|l|c|}
\hline
\textbf{Exigence} & Valeur \\ \hline
Temps de réponse à 5\% & <\SI{40}{ms} \\ \hline
Écart statique & nul \\ \hline
Marge de phase & $\Delta \phi = 60\degres$ \\ \hline
\end{tabular}
\end{center}

La consigne de vitesse $\dot{\theta}_{fe0 cons}(t)= \omega_{fe0 cons}(t)$ est établie par rapport au référentiel galiléen $\rep{0}$. Elle est calculée à partir de la détection de posture  de la tête du pilote et des informations 
d’orientation du porteur par rapport au référentiel terrestre $\rep{0}$ obtenues par la centrale inertielle du porteur.


\begin{center}
\includegraphics[width=\linewidth]{images/fig_02}
%\textit{}

$k_{cfe}=\SI{10,2}{N.A^{-1}}$, $k_{vfe}=\SI{10,2}{V.s.m^{-1}}$, on note 
$K_{fe}=k_{cfe}=k_{vfe}$, $R_{fe}=\SI{7,5}{\Omega}$. 
\end{center}

Dans un premier temps, l’asservissement de vitesse n’est pas corrigé, c’est-à-dire que $H_{cor fe}(p)=1$.


\subparagraph{}\textit{Exprimer littéralement et sous forme canonique la fonction de transfert $H_{fe1}(p)=\dfrac{\Omega_{fe0}(p)}{\Omega_{fe0\text{ cons}}(p)}$, en fonction de $K_1$, $\tau_{gyro}$, $M_{eq}$, $K_{fe}$ et $R_{fe}$.}

\ifprof
\begin{corrige}
\end{corrige}
\else
\fi
Compte tenu des temps de réponse à observer, on montre que $H_{fe1}(p$ peut se mettre sous la forme simplifiée suivante : $H_{fe1}(p)=\dfrac{0,5}{1+3,65\times 10^{-1}p+6\times 10^{-4}p^2}$.
\subparagraph{}\textit{En utilisant l’abaque de la figure suivante, déterminer le temps de réponse à 5\% et l’écart statique de l’asservissement en vitesse de l’étage fin d’élévation en réponse à un échelon de vitesse unitaire. Conclure sur le
respect des performances en rapidité et en précision.}
\ifprof
\begin{corrige}
\end{corrige}
\else
\fi


\begin{center}
\includegraphics[width=\linewidth]{images/fig_03}
%\textit{}
\end{center}


On propose d’utiliser un correcteur proportionnel intégral de la forme $H_{\text{cor }fe}(p)=K_{pfe}\left(1+\dfrac{1}{T_{ife}p}\right)$. La fonction de transfert en boucle ouverte de l’asservissement en vitesse de l’étage fin d’élévation devient alors 
$
H_{BOfe}(p)=K_{pfe}\left( 1+\dfrac{1}{T_{ife}p}\right) \dfrac{1}{1+0,75p} \dfrac{1}{1+1,6\times 10^{-3}p}
$.

La figure *** du document réponse correspond aux tracés des diagrammes de Bode réels de $H_{BOfe}(j\omega)$ 
pour $K_{pfe}=1$ et $T_{ife}=\SI{0,1}{s}$ puis $T_{ife}=\SI{0,01}{s}$.


\subparagraph{}\textit{Sur cette même figure, tracer le diagramme de phase asymptotique de $H_{BOfe}(j\omega)$
(Bode) pour $T_{ife}=\SI{0,1}{s}$, en indiquant la pulsation $\dfrac{1}{T_{ife}}$.}
\ifprof
\begin{corrige}
\end{corrige}
\else
\fi

La lecture du tracé réel de la phase met en évidence un maximum à la pulsation $\omega_{\text{max}}$ telle que $\omega_{\text{max}}\in \left[\dfrac{1}{T_{ife}};600 \right]\text{rad s}^{-1}$.

\subparagraph{}\textit{En supposant que le tracé réel semi-logarithmique de la phase est symétrique autour de $\omega_{\text{max}}$, calculer la valeur de $T_{ife}$ comprise dans la décade $\left[\SI{0,01}{s}; \SI{0,1}{s}\right]$ qui permet de régler ce maximum à -120\degres.}
\ifprof
\begin{corrige}
\end{corrige}
\else
\fi

\subparagraph{}\textit{}
\ifprof
\begin{corrige}
\end{corrige}
\else
\fi

\subparagraph{}\textit{}
\ifprof
\begin{corrige}
\end{corrige}
\else
\fi

\ifprof
\else
\footnotesize

\begin{enumerate}
\item 

\item 
\end{enumerate}
\normalsize
\fi

\ifprof
\else
\end{multicols}
\fi


\end{document}

\subparagraph{}\textit{}
\ifprof
\begin{corrige}
\end{corrige}
\else
\fi

\begin{center}
\includegraphics[width=\linewidth]{images/fig_04}
%\textit{}
\end{center}


