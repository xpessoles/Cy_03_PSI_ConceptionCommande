\documentclass[10pt,fleqn]{article} % Default font size and left-justified equations
\usepackage[%
    pdftitle={Modélisation SLCI : Rapidité des systèmes},
    pdfauthor={Xavier Pessoles}]{hyperref}
    
\input{style/new_style}
\input{style/macros_SII}
\usepackage{multicol}
\usepackage{siunitx}
%\usepackage{picins}
\fichetrue
%\fichefalse

\proftrue
%\proffalse

\tdtrue
%\tdfalse

\courstrue
\coursfalse

\def\discipline{Sciences \\Industrielles de \\ l'Ingénieur}
\def\xxtete{Sciences Industrielles de l'Ingénieur}

\def\classe{PSI$\star$ -- MP}
\def\xxnumpartie{Cycle 02}
\def\xxpartie{Modéliser les systèmes asservis dans le but de prévoir leur comportement}


\def\xxnumchapitre{Chapitre 2 \vspace{.2cm}}
\def\xxchapitre{\hspace{.12cm} Rapidité des systèmes}


\def\xxtitreexo{Modélisation d’un hayon de coffre électrique}
\def\xxsourceexo{\hspace{.2cm} \footnotesize{Sciences industrielles de l'ingénieur MP/MP* PSI/PSI* PT/PT*, Vuibert Prépas -- CCS TSI 2013}}


\def\xxposongletx{2}
\def\xxposonglettext{1.45}
\def\xxposonglety{20}
%\def\xxonglet{Part. 1 -- Ch. 3}
\def\xxonglet{Cycle 02}

\def\xxactivite{TD 01}
\def\xxauteur{\textsl{Éditions Vuibert}}

\def\xxcompetences{%
\textsl{%
\textbf{Savoirs et compétences :}\\
%Les sources sont associées par un \emph{hacheur série}. La détermination des grandeurs électriques associées à ce montage permet de conclure vis à vis du cahier des charges.
%\noindent \textbf{Résoudre :} à partir des modèles retenus :
%\begin{itemize}[label=\ding{112},font=\color{ocre}] 
%\item choisir une méthode de résolution analytique, graphique, numérique;
%\item mettre en \oe{}uvre une méthode de résolution.
%\end{itemize}
%\begin{itemize}[label=\ding{112},font=\color{ocre}] 
%\item \textit{Rés -- C1.1 :} Loi entrée sortie géométrique et cinématique -- Fermeture géométrique.
%\end{itemize}
%
%\noindent \textit{Mod2 -- C4.1 :} Représentation par schéma bloc.
}}

\def\xxfigures{
\includegraphics[width=.5\linewidth]{images/fig_01}
}%figues de la page de garde


\def\xxpied{%
Cycle 01 -- Modéliser le comportement des systèmes multiphysiques\\
Chapitre 2 -- \xxactivite%
}

\setcounter{secnumdepth}{5}
%---------------------------------------------------------------------------

\usepackage{pgfplots}
\begin{document}

%\chapterimage{png/Fond_Cin}
\input{style/new_pagegarde}
\vspace{5cm}
\pagestyle{fancy}
\thispagestyle{plain}

\def\columnseprulecolor{\color{ocre}}
\setlength{\columnseprule}{0.4pt} 

\def\pathfig{images}

\begin{multicols}{2}
Les constructeurs d’automobiles proposent régulièrement des solutions
innovantes pour l’amélioration du confort et de la sécurité des
véhicules. Le hayon de coffre électrique entre dans ce cadre. Développé
par la société Valéo, il permet de manipuler le hayon de coffre sans
effort, depuis l’extérieur ou depuis le siège conducteur.
Le mouvement du hayon doit satisfaire un certain nombre de contraintes,
et en particulier, assurer la sécurité des personnes (mouvement
progressif et effort limité en cas de pincement d’un membre)
et offrir un service adéquat (rapidité et précision suffisante pour un
bon fonctionnement). Le cahier des charges impose les performances
suivantes :
\begin{itemize}
\item stabilité : marge de phase de plus de 45\degres;
\item précision : écart en régime permanent nul vis-à-vis d’une consigne d’angle sous forme d’un
échelon;
\item rapidité : temps de réponse à 5\% de moins de \SI{0,5}{s} ;
\item effort de pincement : \SI{40}{N} maximum 1\% en \SI{10}{ms} maximum.
\end{itemize}
La figure suivante propose une modélisation de l’asservissement de la position angulaire du hayon.

\begin{center}
\includegraphics[width=\linewidth]{images/fig_02}
%\textit{}
\end{center}


Le mouvement angulaire du hayon fait l’objet d’un asservissement. Le schéma-blocs de la
modélise l’architecture choisie et le comportement des composants. Les valeurs numériques
sont les suivantes : 
$L = 1,5.10^{-3} \text{H}$, 
$R = 50.10^{-3}\text{W}$, 
$J = 10^{-4}\text{kg.m}^2$, 
$f = \SI{0,1}{Nms^{-1}}$,
$K_t = 9,5.10^{-3}\text{Nm/A}$, 
$K_r = \SI{0,1}{Nm}$ et $G_c = \SI{2,4}{V/rad}$.


Le correcteur $C(p)$ est dans un premier temps considéré comme unitaire : $C(p) = 1$.

\subparagraph{}\textit{Sans faire aucun calcul, peut-on prévoir, à partir du schéma-blocs, si le
système sera précis et s’il sera sensible aux perturbations ?}

\subparagraph{}\textit{Déterminer la fonction de transfert de la FTBO : $FTBO(p) = \dfrac{\theta(p)}{\varepsilon(p)}$.}

\subparagraph{}\textit{En calculant la limite à convergence pour une entrée consigne en échelon, vérifier si le système est précis.}

\subparagraph{}\textit{Déterminer la fonction de transfert relative à l’entrée en perturbation : $Hr (p) = \dfrac{\theta(p)}{C_r(p)}$.}

\subparagraph{}\textit{Déterminer la limite à convergence pour une perturbation en échelon et vérifier si le système est sensible aux perturbations.}

Après calcul, la fonction de transfert en boucle fermée pour un correcteur unitaire peut s’écrire
sous la forme 
$FTBF(p) = \dfrac{0,82}{1+0,2p+5,6\cdot 10^{-3}p^2+5,4\cdot10^6 p^3}$. 
Les pôles associés sont les suivants : 
$p_1 = \SI{-6,6}{rad.s^{-1}}$, $p_2 = \SI{-28}{rad.s^{-1}}$, 
$p_3 = \SI{-1000}{rad s^{-1}}$.

\subparagraph{}\textit{À partir des informations fournies, peut-on conclure sans calcul sur la précision et la stabilité du système ?}

\subparagraph{}\textit{En adoptant une expression approchée pertinente de la FTBF (par réduction d’ordre), estimer le temps de réponse à 5 \% du système non corrigé.}
\vspace{.25cm}

Le correcteur $C(p)$ est désormais choisi sous la forme $C(p) = K_i\dfrac{1+\tau p}{p}$, 
avec $K_i = \SI{2}{s^{-1}}$ et $\tau =\SI{1}{s}$. Le
diagramme de Bode de la fonction de transfert en boucle ouverte est donné.


\begin{center}
\includegraphics[width=\linewidth]{images/fig_03}
%\textit{}
\end{center}

\subparagraph{}\textit{Le système est-il toujours stable ? Estimer les valeurs des marges de stabilité sur le diagramme
de Bode.}

\subparagraph{}\textit{Comment le correcteur adopté modifie-t-il la précision et la sensibilité aux perturbations ?
Cette conclusion est-elle prévisible sur le diagramme de Bode ?}

\subparagraph{}\textit{Déterminer une estimation de la rapidité du système en boucle fermée à partir de la bande
passante obtenue sur le diagramme de Bode de la FTBO.}

\subparagraph{}\textit{Si un usager se coince la main lors de la fermeture du coffre, comment l’asservissement se
comportera-t-il ? Peut-on déterminer le couple moteur dans ces conditions ?}

\subparagraph{}\textit{Cette valeur du couple moteur conduit à une force de pincement de 120N sur la main. Cette
valeur est-elle acceptable ?}

\end{multicols}


\newpage

\begin{center}
\includegraphics[width=\linewidth]{images/cor_01}
%\textit{}
\end{center}


\begin{center}
\includegraphics[width=\linewidth]{images/cor_02}
%\textit{}
\end{center}


\end{document}

\subparagraph{}\textit{}


\begin{center}
\includegraphics[width=\linewidth]{images/fig_06}
%\textit{}
\end{center}
\begin{center}
\includegraphics[width=\linewidth]{images/img_04}
%\textit{}
\end{center}

