\documentclass[10pt,fleqn]{article} % Derfault font size and left-justified equations
\usepackage[%
    pdftitle={Correction des SLCI : Rapidité des systèmes},
    pdfauthor={Xavier Pessoles}]{hyperref}
    
\input{style/new_style}
\input{style/macros_SII}
\usepackage{multicol}
\usepackage{siunitx}
%\usepackage{picins}
\fichetrue
%\fichefalse

\proftrue
%\proffalse

\tdtrue
%\tdfalse

\courstrue
\coursfalse

\def\discipline{Sciences \\Industrielles de \\ l'Ingénieur}
\def\xxtete{Sciences Industrielles de l'Ingénieur}

\def\classe{PSI$\star$ -- MP}
\def\xxnumpartie{Cycle 03}
\def\xxpartie{Concevoir la partie commande des systèmes asservis afin de valider leurs performances}

\def\xxnumchapitre{Chapitre 1 \vspace{.2cm}}
\def\xxchapitre{\hspace{.12cm} Correction des SLCI}


\def\xxtitreexo{Agitateur médical avec chambre de Riccordi}
\def\xxsourceexo{\hspace{.2cm} \footnotesize{CCP -- PSI -- 2006}}


\def\xxposongletx{2}
\def\xxposonglettext{1.45}
\def\xxposonglety{20}
%\def\xxonglet{Part. 1 -- Ch. 3}
\def\xxonglet{\textsf{Cycle 03}}

\def\xxactivite{TD 01}
\def\xxauteur{\textsl{Xavier Pessoles}}

\def\xxcompetences{%
\textsl{%
\textbf{Savoirs et compétences :}\\
\footnotesize{}%Le modèle du système est donné. Des conditions sont estimées sur un paramètre de la FTBO à partir des exigences du cahier des charges. Une \textbf{démarche de réglage d'un correcteur à avance de phase} est appliquée. Une conclusion est menée en déterminant pour chaque exigence l'écart entre la performance simulée et celle attendue.}
%Les sources sont associées par un \emph{hacheur série}. La détermination des grandeurs électriques associées à ce montage permet de conclure vis à vis du cahier des charges.
%\noindent \textbf{Résoudre :} à partir des modèles retenus :
%\begin{itemize}[label=\ding{112},font=\color{ocre}] 
%\item choisir une méthode de résolution analytique, graphique, numérique;
%\item mettre en \oe{}uvre une méthode de résolution.
%\end{itemize}
%\begin{itemize}[label=\ding{112},font=\color{ocre}] 
%\item \textit{Rés -- C1.1 :} Loi entrée sortie géométrique et cinématique -- Fermeture géométrique.
%\end{itemize}
%
%\noindent \textit{Mod2 -- C4.1 :} Représentation par schéma bloc.
}}

\def\xxfigures{
\includegraphics[width=.8\linewidth]{images/fig_00}
}%figues de la page de garde


\def\xxpied{%
Cycle 03 -- Concevoir la partie commande des SLCI \\
Chapitre 1 -- \xxactivite%
}

\setcounter{secnumdepth}{5}
%---------------------------------------------------------------------------

\usepackage{pgfplots}
\begin{document}

%\chapterimage{png/Fond_Cin}
\input{style/new_pagegarde}
\vspace{5cm}
\pagestyle{fancy}
\thispagestyle{plain}

\def\columnseprulecolor{\color{ocre}}
\setlength{\columnseprule}{0.4pt} 

\def\pathfig{images}

\begin{multicols}{2}

\subsection*{Présentation}

%\noindent\begin{minipage}[c]{.6\linewidth}

Afin d'isoler des cellules issues du pancréas, il est nécessaire de les baigner dans un mélange d'enzymes tout en agitant la solution dans un milieu contrôlé en température. On utilise pour cela 
un agitateur médical avec chambre de Riccordi.


La maîtrise de la température joue un rôle crucial, l’objectif de notre étude est de réduire les temps
de réaction et d’augmenter la précision en température du système de chauffage.
Nous utilisons pour chauffer la solution circulant dans la chambre, un collier chauffant situé sur le
pourtour de la chambre, alimenté en tension par une unité comprenant un correcteur et un
amplificateur.

%Cette unité élabore une tension, dépendant de la tension de consigne fournie par un appareillage
%auxiliaire $U_{tc}$ (non étudié dans cette étude) et de la tension $U_t$ provenant d'un capteur de température
%situé dans la chambre.



On note : 
\begin{itemize}
\item $U_{tc}$ : tension de consigne;
\item $U_t$ : tension à l'image de la température (capteur de température mesurant la température dans la chambre);
\item $U_a$ : tension d'alimentation du collier chauffant;
\item $q_c$ : énergie calorifique fournie par le collier chauffant;
\item $q_p$ : énergie calorifique perdue ou reçue par la chambre (en dehors du collier chauffant) perte par convection, par circulation de l'enzyme.
\end{itemize}

\begin{center}
\includegraphics[width=\linewidth]{images/fig_03}
%\textit{}
\end{center}
%
%\subparagraph{}\textit{Expliquez la signification du sommateur situé, sur le schéma-blocs, en amont du bloc transfert de la chambre.}
%
%
%\subsection*{Identification du système}
%On ouvre la boucle de régulation en ne gardant que l'amplificateur, le collier chauffant, la chambre et le capteur de température.
%
%\begin{center}
%\includegraphics[width=\linewidth]{images/fig_02}
%%\textit{}
%\end{center}
%
%Le système est stabilisé à 17\degres C avec une tension $U_c=\SI{0}{V}$. Le capteur de température est calibré pour
%fournir une tension de \SI{0}{V} à cette température.
%On applique brusquement une tension de \SI{10}{V} à l'entrée de l'amplificateur. On relève à l'aide du
%capteur de température l'augmentation de température (valeur de tension en sortie du capteur de température $U_t$).

Expérimentalement, on peut déterminer que $\text{FTBO(p)}=\dfrac{U_t(p)}{U_c(p)}=\dfrac{0,5}{\left(1+5 p \right)\left(1+100 p \right)}$.

\subsection*{Analyse des performances}
On considère ici que $C(p)=1$. On donne l'abaque des temps de réponse réduit plus bas.

\subparagraph{}\textit{Déterminer le temps de réponse à 5\% du système régulé.}
\subparagraph{}\textit{Déterminer l'écart en position et l'écart en traînage.}

\subparagraph{}\textit{Tracer le diagramme de Bode de la FTBO.}

\subparagraph{}\textit{Déterminer la marge de gain et la marge de phase.}

\subsection*{Mise en \oe{}uvre de corrections P et PI}

On envisage une première correction en utilisant un correcteur proportionnel de la forme $C(p)=K$.


\subparagraph{}\textit{Déterminer le gain $K$ de manière à obtenir le système le plus rapide sans aucun dépassement.}


\subparagraph{}\textit{En déduire le temps de réponse à 5\%, l'écart en position et l'écart de traînage.}

\subparagraph{}\textit{Déterminez alors, la tension en entrée de l'amplificateur , si on envoie un échelon de tension de consigne $U_{tc}$ de \SI{5}{V}. Le gain de l'amplificateur étant de 10, critiquez vos résultats.}

On souhaite maintenant corriger le système avec en utilisant une action proportionnelle intégrale $C(p)=\dfrac{K}{T_i p}\left( 1+T_i p\right)$. On souhaite que la montée en température soit de 3 minutes maximum. 

\subparagraph{}\textit{Déterminer les gain $K$ et $T_i$ permettant d'assurer le non dépassement de la consigne ainsi que le temps de réponses du système.}


\subparagraph{}\textit{En déduire le nouvel écart de position.}
%\subparagraph{}\textit{\'A l'aide du tracé expérimental du document réponse « Réponse indicielle et Zoom »
%qui correspond à l'augmentation de tension $U_t$ en sortie du capteur de température, identifier la
%fonction de transfert du système en Boucle ouverte défini par le rapport $\text{FTBO(p)}=\dfrac{U_t(p)}{U_c(p)}$. On pourra utiliser la forme suivante : $\dfrac{G}{\left(1+\tau_1 p \right)\left(1+\tau_2 p \right)}$.}

\begin{center}
\includegraphics[width=\linewidth]{images/fig_04}
%%\textit{}
\end{center}


\end{multicols}
\newpage

\begin{center}
\includegraphics[width=\linewidth]{images/cor_01}
%%\textit{}
\end{center}
\begin{center}
\includegraphics[width=\linewidth]{images/cor_02}
%%\textit{}
\end{center}
\begin{center}
\includegraphics[width=\linewidth]{images/cor_03}
%%\textit{}
\end{center}
\begin{center}
\includegraphics[width=\linewidth]{images/cor_04}
%%\textit{}
\end{center}
\begin{center}
\includegraphics[width=\linewidth]{images/cor_05}
%%\textit{}
\end{center}
\begin{center}
\includegraphics[width=\linewidth]{images/cor_06}
%%\textit{}
\end{center}
\begin{center}
\includegraphics[width=\linewidth]{images/cor_07}
%%\textit{}
\end{center}


%
%\begin{center}
%\includegraphics[width=\linewidth]{images/fig_07}
%%\textit{}
%\end{center}


\end{document}

\subparagraph{}\textit{}
\begin{center}
\includegraphics[width=\linewidth]{images/fig_07}
%\textit{}
\end{center}


%\newpage

%\begin{center}
%\includegraphics[width=\linewidth]{images/cor_01}
%\textit{}
%\end{center}






\begin{center}
\includegraphics[width=\linewidth]{images/fig_06}
%\textit{}
\end{center}
\begin{center}
\includegraphics[width=\linewidth]{images/img_04}
%\textit{}
\end{center}

