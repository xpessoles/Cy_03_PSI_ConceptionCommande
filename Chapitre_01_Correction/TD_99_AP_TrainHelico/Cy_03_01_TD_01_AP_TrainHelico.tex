\documentclass[10pt,fleqn]{article} % Derfault font size and left-justified equations
\usepackage[%
    pdftitle={Correction des SLCI : Correcteur Avance de Phase},
    pdfauthor={Xavier Pessoles}]{hyperref}

\input{style/new_style}
\input{style/macros_SII}
\usepackage{multicol}
\usepackage{siunitx}
\usepackage{schemabloc}
%\usepackage{picins}
\fichetrue
%\fichefalse

\proftrue
\proffalse

\tdtrue
%\tdfalse

\courstrue
\coursfalse

\newif\ifnormal
\normaltrue
%\normalfalse

\newif\ifdifficile
\difficilefalse
%\difficiletrue

\newif\iftdifficile
\tdifficilefalse
%\tdifficiletrue

% -------------------------------------
% Déclaration des titres
% -------------------------------------

\def\classe{\textsf{PSI$\star$ -- MP}}
\def\xxnumpartie{Cycle 03}
\def\xxpartie{Concevoir la partie commande des systèmes asservis afin de valider leurs performances}


\def\xxnumchapitre{Chapitre 1 \vspace{.2cm}}
\def\xxchapitre{\hspace{.12cm} Correction des SLCI}

\def\discipline{Sciences \\Industrielles de \\ l'Ingénieur}
\def\xxtete{Sciences Industrielles de l'Ingénieur}

\def\xxposongletx{2}
\def\xxposonglettext{1.45}
\def\xxposonglety{19}%16

\def\xxonglet{\textsf{Cycle 03}}

\def\xxactivite{TD 99}
\def\xxauteur{\textsl{Xavier Pessoles}}


\def\xxtitreexo{Train d'aterrissage d'hélicoptère \ifnormal $\star$ \else \fi \iftdifficile $\star\star\star$ \else \fi }
\def\xxsourceexo{\hspace{.2cm} \footnotesize{Banque PT -- SIA 2014}}

\def\xxcompetences{%
\textsl{%
\textbf{Savoirs et compétences :}\\
}}

\def\xxfigures{
\includegraphics[width=.65\textwidth]{images/fig_00}
}%figues de la page de garde
\def\xxpied{%
Cycle 03 -- Concevoir la commande des SLCI\\% afin de valider leurs performances.\\
Chapitre 1 -- \xxactivite%
}


\setcounter{secnumdepth}{5}
%---------------------------------------------------------------------------


\begin{document}
%\chapterimage{png/Fond_Cin}
\input{style/new_pagegarde}
\vspace{4.5cm}
\pagestyle{fancy}
\thispagestyle{plain}


\def\columnseprulecolor{\color{ocre}}
\setlength{\columnseprule}{0.4pt} 

\ifprof
%\begin{multicols}{2}
\else
\begin{multicols}{2}
\fi


\section*{Mise en situation}
\begin{obj}
Cette partie a pour objectif de proposer un réglage pour le correcteur de la commande asservie définie dans la partie précédente. Il s'agira également de valider les performances globales obtenues grâce à cette commande semi-active.
\end{obj}

On se propose d'étudier la stabilité vis-à-vis de la seule consigne  $\dot{Z}_c^*(p)$. On adopter pour le réglage de la correction le schéma suivant.

\begin{center}
\includegraphics[width=\linewidth]{images/fig_01}
%\textit{}
\end{center}

On note dans ce schéma :
\begin{itemize}
\item $\dot{Z}^*(p)$ la transformée de $\dot{z}^*(t)=\dot{z}(t)+V_0$  avec $V_0$ la vitesse d'impact et $\dot{z}(t)$ la vitesse absolue de la cabine par rapport au sol;% (voir figure 8 annexe 5).
\item $F_{\text{éq}}(p)$  l'effort équivalent ramené au déplacement de la cabine et fourni par la partie active de l'amortisseur;% (voir partie D1).
\item $\lambda_a$ le coefficient d'amortissement passif équivalent ramené au déplacement de la cabine;
\item $H_S(p)=\dfrac{K_S}{1+T_Sp}$ la fonction de transfert de la partie active de l'amortisseur. %Indépendamment de la partie précédente, 
On prendra : $K_S = \SI{12e4}{N.A^{-1}}$ et $T_S = \SI{5e-3}{s}$;
\item $H_Z(p)=\dfrac{K_Z p^2}{1+\dfrac{2\xi_Z}{\omega_Z}p+\dfrac{p^2}{\omega_Z^2}}$ la fonction de transfert traduisant le comportement dynamique du train.
\item $C(p)$ la fonction de transfert du correcteur dont le réglage fait l'objet de cette partie.
\end{itemize}
%$\dot{Z}_c^*(p)$  est la transformée de la consigne  $\dot{z}_c^*(t)$. On donne ci-dessous l'allure de la consigne en vitesse avec une accélération de $\SI{12}{m.s^{-2}}$ et une vitesse maximale de $\SI{4}{m.s^{-1}}$. 
%
%
%\begin{center}
%\includegraphics[width=.5\linewidth]{images/fig_02}
%%\textit{}
%\end{center}

\subsection*{Fonction de transfert en boucle ouverte non corrigée}
\begin{obj}
II s'agit dans un premier temps d'analyser la forme de la fonction de transfert en boucle ouverte non corrigée de la chaîne de commande semi-active.
\end{obj}

\subparagraph{}\textit{Déterminer littéralement et sous forme canonique la fonction de transfert  $H_F(p)=\dfrac{\dot{Z}^*(p)}{F_{\text{éq}}(p)}$.}
\ifprof
\begin{corrige}
\end{corrige}
\else
\fi

\subparagraph{}\textit{Déterminer littéralement et sous forme canonique la fonction de transfert en boucle ouverte non corrigée $H_{\text{BONC}}(p)$.}
\ifprof
\begin{corrige}
\end{corrige}
\else
\fi

On donne le diagramme de Bode de $H_F(p)$.

\begin{center}
\includegraphics[width=.9\linewidth]{images/fig_03}
%\textit{}
\end{center}

\subparagraph{}\textit{Justifier la forme de ce diagramme en traçant les asymptotes et en indiquant comment retrouver sur le tracé les valeurs de $K_z$ et $\omega_z$. Tracer en rouge les diagrammes de la fonction $H_{\text{BONC}}(p)$. On prendra pour cela $20\log K_S \simeq \SI{100}{dB}$.}
\ifprof
\begin{corrige}
\end{corrige}
\else
\fi

\subsection*{Choix et réglage de la correction}
\begin{obj}
II s'agit à présent de définir la structure du correcteur et de proposer un réglage permettant de satisfaire les critères du cahier des charges.
\end{obj}

Afin de satisfaire les exigences, une étude complémentaire non abordée dans ce sujet montre que la boucle d'asservissement doit posséder les performances suivantes :
\begin{itemize}
\item erreur statique nulle
\item pulsation de coupure à \SI{0}{dB} et $\omega_{\SI{0}{dB}}= \SI{6}{rad.s^{-1}}$;
\item marge de phase $M\varphi = 45\degres$;
\item marge de gain $MG > \SI{6}{dB}$.
\end{itemize}


\subparagraph{}\textit{Quelle doit être la classe minimale du correcteur afin de garantir le critère de précision ?}
\ifprof
\begin{corrige}
\end{corrige}
\else
\fi

On choisit dans un premier temps un correcteur de la forme $C(p)=\dfrac{K_p}{p^2}$. On donne les diagrammes de Bode de la fonction de transfert en boucle ouverte du système ainsi corrigé pour $K_p =1$. 

\begin{center}
\includegraphics[width=.9\linewidth]{images/fig_04}
%\textit{}
\end{center}

\subparagraph{}\textit{Évaluer les marges de stabilité pour ce réglage. Déterminer la valeur de $K_p$ garantissant le critère de pulsation de coupure à \SI{0}{dB}. Ce correcteur peut-il permettre de répondre aux critères de performances énoncés en début de partie ? Justifier la réponse}
\ifprof
\begin{corrige}
\end{corrige}
\else
\fi

On choisit finalement un correcteur de la forme $C(p)= \dfrac{K_p}{p^2}\dfrac{1+\mu Tp}{1+Tp}$ avec $\mu > 1$.
Les caractéristiques du terme en $K_p\dfrac{1+\mu Tp}{1+Tp}$  ainsi que des abaques de calcul sont donnés ci -dessous. On reprend dans un premier temps $K_p=1$. 

\begin{center}
\includegraphics[width=\linewidth]{images/fig_05}
%\textit{}
\end{center}

\subparagraph{}\textit{Comment se nomme l'action de correction obtenue avec ce terme ? }
\ifprof
\begin{corrige}
\end{corrige}
\else
\fi


\subparagraph{}\textit{Quelle valeur doit-on donner à $\mu$ pour garantir le critère de marge de phase ? }
\ifprof
\begin{corrige}
\end{corrige}
\else
\fi


\subparagraph{}\textit{En déduire les valeurs de $T$ et de $K_P$ permettant d'assurer les critères de stabilité et de bande passante énoncés au début de partie. Le critère de précision est-il validé ?}
\ifprof
\begin{corrige}
\end{corrige}
\else
\fi

\subsection*{Validation des performances}

\begin{obj}
II s'agit dans cette dernière partie de vérifier les performances globales de la boucle d'asservissement.
\end{obj}

On donne le résultat d'une simulation du système complet piloté à l'aide du correcteur précédemment dimensionné pour une vitesse d'impact de $\SI{4}{m.s^{-1}}$.

\subparagraph{}\textit{En analysant cette courbe, conclure quant à la validité du cahier des charges.}
\ifprof
\begin{corrige}
\end{corrige}
\else
\fi


\begin{center}
\includegraphics[width=\linewidth]{images/fig_06}
%\textit{}
\end{center}






\ifprof
\else
\end{multicols}
\fi


\end{document}

\subparagraph{}\textit{}
\ifprof
\begin{corrige}
\end{corrige}
\else
\fi

\begin{center}
\includegraphics[width=\linewidth]{images/fig_04}
%\textit{}
\end{center}


